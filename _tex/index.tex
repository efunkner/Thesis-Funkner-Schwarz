% Options for packages loaded elsewhere
% Options for packages loaded elsewhere
\PassOptionsToPackage{unicode,linktoc=all}{hyperref}
\PassOptionsToPackage{hyphens}{url}
\PassOptionsToPackage{dvipsnames,svgnames,x11names}{xcolor}
%
\documentclass[
  ngerman,
  letterpaper,
  DIV=11]{scrreprt}
\usepackage{xcolor}
\usepackage[top=30mm,left=20mm,heightrounded]{geometry}
\usepackage{amsmath,amssymb}
\setcounter{secnumdepth}{-\maxdimen} % remove section numbering
\usepackage{iftex}
\ifPDFTeX
  \usepackage[T1]{fontenc}
  \usepackage[utf8]{inputenc}
  \usepackage{textcomp} % provide euro and other symbols
\else % if luatex or xetex
  \usepackage{unicode-math} % this also loads fontspec
  \defaultfontfeatures{Scale=MatchLowercase}
  \defaultfontfeatures[\rmfamily]{Ligatures=TeX,Scale=1}
\fi
\usepackage{lmodern}
\ifPDFTeX\else
  % xetex/luatex font selection
\fi
% Use upquote if available, for straight quotes in verbatim environments
\IfFileExists{upquote.sty}{\usepackage{upquote}}{}
\IfFileExists{microtype.sty}{% use microtype if available
  \usepackage[]{microtype}
  \UseMicrotypeSet[protrusion]{basicmath} % disable protrusion for tt fonts
}{}
\makeatletter
\@ifundefined{KOMAClassName}{% if non-KOMA class
  \IfFileExists{parskip.sty}{%
    \usepackage{parskip}
  }{% else
    \setlength{\parindent}{0pt}
    \setlength{\parskip}{6pt plus 2pt minus 1pt}}
}{% if KOMA class
  \KOMAoptions{parskip=half}}
\makeatother
% Make \paragraph and \subparagraph free-standing
\makeatletter
\ifx\paragraph\undefined\else
  \let\oldparagraph\paragraph
  \renewcommand{\paragraph}{
    \@ifstar
      \xxxParagraphStar
      \xxxParagraphNoStar
  }
  \newcommand{\xxxParagraphStar}[1]{\oldparagraph*{#1}\mbox{}}
  \newcommand{\xxxParagraphNoStar}[1]{\oldparagraph{#1}\mbox{}}
\fi
\ifx\subparagraph\undefined\else
  \let\oldsubparagraph\subparagraph
  \renewcommand{\subparagraph}{
    \@ifstar
      \xxxSubParagraphStar
      \xxxSubParagraphNoStar
  }
  \newcommand{\xxxSubParagraphStar}[1]{\oldsubparagraph*{#1}\mbox{}}
  \newcommand{\xxxSubParagraphNoStar}[1]{\oldsubparagraph{#1}\mbox{}}
\fi
\makeatother


\usepackage{longtable,booktabs,array}
\usepackage{calc} % for calculating minipage widths
% Correct order of tables after \paragraph or \subparagraph
\usepackage{etoolbox}
\makeatletter
\patchcmd\longtable{\par}{\if@noskipsec\mbox{}\fi\par}{}{}
\makeatother
% Allow footnotes in longtable head/foot
\IfFileExists{footnotehyper.sty}{\usepackage{footnotehyper}}{\usepackage{footnote}}
\makesavenoteenv{longtable}
\usepackage{graphicx}
\makeatletter
\newsavebox\pandoc@box
\newcommand*\pandocbounded[1]{% scales image to fit in text height/width
  \sbox\pandoc@box{#1}%
  \Gscale@div\@tempa{\textheight}{\dimexpr\ht\pandoc@box+\dp\pandoc@box\relax}%
  \Gscale@div\@tempb{\linewidth}{\wd\pandoc@box}%
  \ifdim\@tempb\p@<\@tempa\p@\let\@tempa\@tempb\fi% select the smaller of both
  \ifdim\@tempa\p@<\p@\scalebox{\@tempa}{\usebox\pandoc@box}%
  \else\usebox{\pandoc@box}%
  \fi%
}
% Set default figure placement to htbp
\def\fps@figure{htbp}
\makeatother



\ifLuaTeX
\usepackage[bidi=basic]{babel}
\else
\usepackage[bidi=default]{babel}
\fi
% get rid of language-specific shorthands (see #6817):
\let\LanguageShortHands\languageshorthands
\def\languageshorthands#1{}
\ifLuaTeX
  \usepackage[german]{selnolig} % disable illegal ligatures
\fi


\setlength{\emergencystretch}{3em} % prevent overfull lines

\providecommand{\tightlist}{%
  \setlength{\itemsep}{0pt}\setlength{\parskip}{0pt}}



 


\KOMAoption{captions}{tableheading}
\makeatletter
\@ifpackageloaded{caption}{}{\usepackage{caption}}
\AtBeginDocument{%
\ifdefined\contentsname
  \renewcommand*\contentsname{Inhaltsverzeichnis}
\else
  \newcommand\contentsname{Inhaltsverzeichnis}
\fi
\ifdefined\listfigurename
  \renewcommand*\listfigurename{Abbildungsverzeichnis}
\else
  \newcommand\listfigurename{Abbildungsverzeichnis}
\fi
\ifdefined\listtablename
  \renewcommand*\listtablename{Tabellenverzeichnis}
\else
  \newcommand\listtablename{Tabellenverzeichnis}
\fi
\ifdefined\figurename
  \renewcommand*\figurename{Abbildung}
\else
  \newcommand\figurename{Abbildung}
\fi
\ifdefined\tablename
  \renewcommand*\tablename{Tabelle}
\else
  \newcommand\tablename{Tabelle}
\fi
}
\@ifpackageloaded{float}{}{\usepackage{float}}
\floatstyle{ruled}
\@ifundefined{c@chapter}{\newfloat{codelisting}{h}{lop}}{\newfloat{codelisting}{h}{lop}[chapter]}
\floatname{codelisting}{Listing}
\newcommand*\listoflistings{\listof{codelisting}{Listingverzeichnis}}
\makeatother
\makeatletter
\makeatother
\makeatletter
\@ifpackageloaded{caption}{}{\usepackage{caption}}
\@ifpackageloaded{subcaption}{}{\usepackage{subcaption}}
\makeatother
\usepackage{bookmark}
\IfFileExists{xurl.sty}{\usepackage{xurl}}{} % add URL line breaks if available
\urlstyle{same}
\hypersetup{
  pdftitle={Lehrdemonstrationen zur digitalen Filterung auf eingebetteten Systemen},
  pdfauthor={Eduard Funkner; Martin Schwarz},
  pdflang={de},
  colorlinks=true,
  linkcolor={blue},
  filecolor={Maroon},
  citecolor={Blue},
  urlcolor={Blue},
  pdfcreator={LaTeX via pandoc}}


\title{Lehrdemonstrationen zur digitalen Filterung auf eingebetteten
Systemen}
\author{Eduard Funkner \and Martin Schwarz}
\date{2025-05-28}
\begin{document}
\maketitle

\renewcommand*\contentsname{Inhaltsverzeichnis}
{
\hypersetup{linkcolor=}
\setcounter{tocdepth}{2}
\tableofcontents
}

\chapter{Einführung}\label{einfuxfchrung}

In der Welt der Elektrotechnik werden Signale verschiedener Arten
gesendet und empfangen. Es wird zwischen kontinuierlichen analogen
Signalen und diskretisierten digitalen Signalen differenziert. Ein
analoges Signal könnte eine Tonspur, welche von einem Mikrofon
aufgenommen wurde sein. Bei digitalen Signalen werden Bitfolgen gesendet
die Information enthalten. Diese Bitfolge könnte eine E-Mail sein.

Um diese Signale nutzen zu können müssen diese zuvor gefiltert und
verarbeitet werden. Die Filterung dieser Signale erfolgt durch digitale
oder analoge Filter. Abhängig von der Anwendung werden spezifischen
Filter implementiert. Die Frequenzbereiche der Signale können Mittels
dieser Filter verstärkt oder gedämpft werden damit diese dann
weiterverarbeitet werden können.

\chapter{Zielsetzung und Vorgehen}\label{zielsetzung-und-vorgehen}

\begin{verbatim}
 Implementierung digitaler Filter auf DSP
 Lerndemonstration von der digitalen Signalverarbeitung auf DSP
\end{verbatim}

\chapter{Signalverarbeitung auf Microcontroller
(ESP32)}\label{signalverarbeitung-auf-microcontroller-esp32}

Zur Implementierung der digitalen Filter, auf Mikrocontrollern, wird ein
ESP32 verwendet.

Spezifisch für Audioanwendungen werden Audioboards, wie das ESP Lyrat
Mini oder ESP Lyrat 4.3 genutzt. Diese Boards verfügen über Audiocodecs,
welche analoge Signale in ein digitales Format konvertieren können. Die
Datenübertragung zwischen den Codecs und dem ESP32 geschiet über I2S.
Dieser I2S Datenstrom kann auf dem ESP32 verarbeitet und auf der SD
Karte der Board gespeichert werden. Über diese Audioboards soll eine
analoges Signal in Form von Sprache aufgenommen werden, von dem ESP32
mit den digitalen Filtern gefiltert werden und anschließend abgespielt,
oder auf einer SD Karte gespeichert werden. Außerdem sollen Audio
Dateien von einer SD ausgelesen, gefiltert und abgespielt werden.

Der Audiofilter (Biquad) wird aus dem Skript Digital Signal Processing
Lecuture von S. Spors entnommen und in Arduino, sowie Micropython auf
dem ESP32 implementiert.

Zur Hilfe werden von P.Schatzmann die GitHub Repositorien für die Audio
Tools sowie Audio Driver genutzt, um die Implementierung in Arduino
durchzuführen.

\chapter{Signalverabreitung auf FPGA
(PYNQ)}\label{signalverabreitung-auf-fpga-pynq}

Bevor ein Filter in Hardware implementiert werden kann, muss er zunächst
entworfen werden. Dabei ist zu entscheiden, welche Art digitaler Filter
zum Einsatz kommen soll. Für Audioanwendungen eignen sich insbesondere
linearphasige Filter. Besonders FIR-Filter werden häufig verwendet, da
sie eine konstante Gruppenlaufzeit aufweisen.

IIR-Filter bieten zwar eine kürzere Gruppenverzögerung, sind jedoch
aufgrund ihrer nicht konstanten Gruppenlaufzeit anfälliger für
Verzerrungen. Zudem sind FIR-Filter in der Regel einfacher zu
realisieren. Ein FIR-Filter kann entweder mit Tools wie dem fdatool in
MATLAB entworfen oder aus einem Prototypenfilter abgeleitet werden,
häufig aus einem IIR-Filter mit den gewünschten Eigenschaften.

Im nächsten Schritt wird das PYNQ-Z2 vorbereitet. Dazu muss die
erforderliche Software auf einem PC installiert werden, einschließlich
aller notwendigen Bibliotheken und Module für die Programmierung sowohl
mit VHDL.

Anschließend wird das Board in Betrieb genommen, um sich mit seiner
Funktionsweise vertraut zu machen. Dazu wird erster Testcode
aufgespielt. Dieser Schritt dient dazu, ein grundlegendes Verständnis
der Hardware zu erlangen und eine bessere Vorstellung der
Umsetzungsmöglichkeiten zu bekommen.

Danach wird zunächst ein Audiosignal in das Board eingespeist und ohne
Filterung wieder ausgegeben. Dies dient dazu, sich mit dem Board
vertraut zu machen und sicherzustellen, dass die Einspeisung und Ausgabe
von Audio grundlegend beherrscht wird.

Anschließend werden die Filter entworfen. (Mit dem fdatool in MATLAB
lassen sich die Filter auswahl des Typens und den Koeffizeinten direkt
für andere Anwendungen wie VHDL umsetzten)

\section{Jupyter Notebooks}\label{jupyter-notebooks}

Das PYNQ-Z2 verfügt über ein eigenes Betriebssystem, das zunächst
eingerichtet werden muss. Dieses enthält eine Jupyter-Notebook-Umgebung,
über die das Board konfiguriert und dokumentiert werden kann. Zudem sind
Beispielprojekte verfügbar. Der Zugriff auf das Board erfolgt remote
über die IP-Adresse. Das Dateisystem ist über denselben Weg im
Datei-Explorer erreichbar. Zusätzlich steht ein Jupyter-Terminal zur
Verfügung. Um mit dem Laptop arbeiten zu können, sollte das Board
entsprechend eingerichtet werden. Es empfiehlt sich, den Setup-Guide
vollständig durchzugehen.

\section{VHDL}\label{vhdl}

In VHDL müssen alle erforderlichen Top-Module für das Board entweder
geschrieben oder bestehende Module eingebunden werden. Dabei wird unter
anderem die Portzuweisung vorgenommen. Anschließend wird der Filter
selbst in VHDL implementiert und mithilfe von VIVADO synthetisiert. Nach
der Synthese wird der Bitstream für das Board generiert.




\end{document}

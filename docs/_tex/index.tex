% Options for packages loaded elsewhere
% Options for packages loaded elsewhere
\PassOptionsToPackage{unicode,linktoc=all}{hyperref}
\PassOptionsToPackage{hyphens}{url}
\PassOptionsToPackage{dvipsnames,svgnames,x11names}{xcolor}
%
\documentclass[
  ngerman,
  letterpaper,
  DIV=11]{scrreprt}
\usepackage{xcolor}
\usepackage[top=30mm,left=20mm,heightrounded]{geometry}
\usepackage{amsmath,amssymb}
\setcounter{secnumdepth}{-\maxdimen} % remove section numbering
\usepackage{iftex}
\ifPDFTeX
  \usepackage[T1]{fontenc}
  \usepackage[utf8]{inputenc}
  \usepackage{textcomp} % provide euro and other symbols
\else % if luatex or xetex
  \usepackage{unicode-math} % this also loads fontspec
  \defaultfontfeatures{Scale=MatchLowercase}
  \defaultfontfeatures[\rmfamily]{Ligatures=TeX,Scale=1}
\fi
\usepackage{lmodern}
\ifPDFTeX\else
  % xetex/luatex font selection
\fi
% Use upquote if available, for straight quotes in verbatim environments
\IfFileExists{upquote.sty}{\usepackage{upquote}}{}
\IfFileExists{microtype.sty}{% use microtype if available
  \usepackage[]{microtype}
  \UseMicrotypeSet[protrusion]{basicmath} % disable protrusion for tt fonts
}{}
\makeatletter
\@ifundefined{KOMAClassName}{% if non-KOMA class
  \IfFileExists{parskip.sty}{%
    \usepackage{parskip}
  }{% else
    \setlength{\parindent}{0pt}
    \setlength{\parskip}{6pt plus 2pt minus 1pt}}
}{% if KOMA class
  \KOMAoptions{parskip=half}}
\makeatother
% Make \paragraph and \subparagraph free-standing
\makeatletter
\ifx\paragraph\undefined\else
  \let\oldparagraph\paragraph
  \renewcommand{\paragraph}{
    \@ifstar
      \xxxParagraphStar
      \xxxParagraphNoStar
  }
  \newcommand{\xxxParagraphStar}[1]{\oldparagraph*{#1}\mbox{}}
  \newcommand{\xxxParagraphNoStar}[1]{\oldparagraph{#1}\mbox{}}
\fi
\ifx\subparagraph\undefined\else
  \let\oldsubparagraph\subparagraph
  \renewcommand{\subparagraph}{
    \@ifstar
      \xxxSubParagraphStar
      \xxxSubParagraphNoStar
  }
  \newcommand{\xxxSubParagraphStar}[1]{\oldsubparagraph*{#1}\mbox{}}
  \newcommand{\xxxSubParagraphNoStar}[1]{\oldsubparagraph{#1}\mbox{}}
\fi
\makeatother


\usepackage{longtable,booktabs,array}
\usepackage{calc} % for calculating minipage widths
% Correct order of tables after \paragraph or \subparagraph
\usepackage{etoolbox}
\makeatletter
\patchcmd\longtable{\par}{\if@noskipsec\mbox{}\fi\par}{}{}
\makeatother
% Allow footnotes in longtable head/foot
\IfFileExists{footnotehyper.sty}{\usepackage{footnotehyper}}{\usepackage{footnote}}
\makesavenoteenv{longtable}
\usepackage{graphicx}
\makeatletter
\newsavebox\pandoc@box
\newcommand*\pandocbounded[1]{% scales image to fit in text height/width
  \sbox\pandoc@box{#1}%
  \Gscale@div\@tempa{\textheight}{\dimexpr\ht\pandoc@box+\dp\pandoc@box\relax}%
  \Gscale@div\@tempb{\linewidth}{\wd\pandoc@box}%
  \ifdim\@tempb\p@<\@tempa\p@\let\@tempa\@tempb\fi% select the smaller of both
  \ifdim\@tempa\p@<\p@\scalebox{\@tempa}{\usebox\pandoc@box}%
  \else\usebox{\pandoc@box}%
  \fi%
}
% Set default figure placement to htbp
\def\fps@figure{htbp}
\makeatother



\ifLuaTeX
\usepackage[bidi=basic]{babel}
\else
\usepackage[bidi=default]{babel}
\fi
% get rid of language-specific shorthands (see #6817):
\let\LanguageShortHands\languageshorthands
\def\languageshorthands#1{}
\ifLuaTeX
  \usepackage[german]{selnolig} % disable illegal ligatures
\fi


\setlength{\emergencystretch}{3em} % prevent overfull lines

\providecommand{\tightlist}{%
  \setlength{\itemsep}{0pt}\setlength{\parskip}{0pt}}



 


\KOMAoption{captions}{tableheading}
\makeatletter
\@ifpackageloaded{caption}{}{\usepackage{caption}}
\AtBeginDocument{%
\ifdefined\contentsname
  \renewcommand*\contentsname{Inhaltsverzeichnis}
\else
  \newcommand\contentsname{Inhaltsverzeichnis}
\fi
\ifdefined\listfigurename
  \renewcommand*\listfigurename{Abbildungsverzeichnis}
\else
  \newcommand\listfigurename{Abbildungsverzeichnis}
\fi
\ifdefined\listtablename
  \renewcommand*\listtablename{Tabellenverzeichnis}
\else
  \newcommand\listtablename{Tabellenverzeichnis}
\fi
\ifdefined\figurename
  \renewcommand*\figurename{Abbildung}
\else
  \newcommand\figurename{Abbildung}
\fi
\ifdefined\tablename
  \renewcommand*\tablename{Tabelle}
\else
  \newcommand\tablename{Tabelle}
\fi
}
\@ifpackageloaded{float}{}{\usepackage{float}}
\floatstyle{ruled}
\@ifundefined{c@chapter}{\newfloat{codelisting}{h}{lop}}{\newfloat{codelisting}{h}{lop}[chapter]}
\floatname{codelisting}{Listing}
\newcommand*\listoflistings{\listof{codelisting}{Listingverzeichnis}}
\makeatother
\makeatletter
\makeatother
\makeatletter
\@ifpackageloaded{caption}{}{\usepackage{caption}}
\@ifpackageloaded{subcaption}{}{\usepackage{subcaption}}
\makeatother
\usepackage{bookmark}
\IfFileExists{xurl.sty}{\usepackage{xurl}}{} % add URL line breaks if available
\urlstyle{same}
\hypersetup{
  pdftitle={Lehrdemonstrationen zur digitalen Filterung auf eingebetteten Systemen},
  pdfauthor={Eduard Funkner; Martin Schwarz},
  pdflang={de},
  colorlinks=true,
  linkcolor={blue},
  filecolor={Maroon},
  citecolor={Blue},
  urlcolor={Blue},
  pdfcreator={LaTeX via pandoc}}


\title{Lehrdemonstrationen zur digitalen Filterung auf eingebetteten
Systemen}
\author{Eduard Funkner \and Martin Schwarz}
\date{2025-06-19}
\begin{document}
\maketitle

\renewcommand*\contentsname{Inhaltsverzeichnis}
{
\hypersetup{linkcolor=}
\setcounter{tocdepth}{2}
\tableofcontents
}

\section{Einführung}\label{einfuxfchrung}

Diese Bachlorthesis dient als eine Lerndemonstration zur Implentierung
von digitalen biquadratischen-Filtern auf Mikrocontrollern sowie FPGAs.
Im Umfang dieser Lerndemonstration werden digitale Biquad-Filter
mithilfe von Matlab- und Pythontools entworfen und im Anschluss auf Hard
und Software umgesetzt. Der Prozess der Umsetzung wird zum Verständis
dokumentiert sowie genau erklärt, damit Leser dieser Lerndemonstration
mithilfe ihrer Hardware für sich den Prozess reproduzieren können.

\section{IIR-Filter zweiter Ordnung}\label{iir-filter-zweiter-ordnung}

Infinite Impulse Response (IIR) Filter gehören in der Signalverarbeitung
zu den grundlegenden Typen der digitalen Filter. Grundlegend zeichnen
sich IIR-Filter durch ihre characteristik aus, bei der Berechnung des
aktuellen Ausgangssingals durch die Verwendung des gegenwärtigen und
vergangenen Eingabewertes als auch des vorherigen Ausgabewertes.
Aufgrund dieser Rückkopplung kann es dazu führen, dass die Impulsantwort
von IIR-Filtern theoretisch unendlich lang andauern kann, was sie
grundsätzlich von Finite Impulse Response (FIR) Filtern unterscheidet.

Ein immenser Vorteil von IIR-Filtern liegt in der hohen Effizienz bei
der Realisierung scharfter Frequenzgänge mit relativ wenigen
Koeffizienten, während FIR-Filter für vergleichbare Filtereigenschaften
oft hunderte von Koeffizienten benötigen, können IIR-Filter ähnliche
Ergebnisse mit deutlich geringerem Rechenaufwand erreichen. Aufgrund
dieser Eigenschaft eignen sich IIR-Filter besonders gut im Einsatz von
Systemen mit beschränkten Ressourcen oder in Echtzeitverarbeitung.

Biquadratische Filter, kurz als ``Biquad'' bezeichnet, sind eine
spezielle Unterklasse von IIR-Filtern zweiter Ordnung. Durch das
Kaskadieren solcher Biquad-Sektionen besteht die möglichkeit komplexe
IIR-Filter höhrer Ordnung auf einfache Weise zu realsieren. Das
Kaskadieren dieser Sektionen bringt Vorteile in Bezug auf numerische
Stabilität, Flexibilät bei der Parameteranpassung und Modularität des
Designs.

\chapter{Differenzengleichung eines IIR Filters zweiter
Ordnung}\label{differenzengleichung-eines-iir-filters-zweiter-ordnung}

Digitale Filter können mit einer linearen Differenzengleichung
beschrieben werden. Differenzengleichung des Biquad-Filters:

\[
y[n] =  \frac{1}{a_0}(b_0 \cdot x[n] + b_1 \cdot x[n - 1] + b_2 \cdot x[n - 2] - a_1 \cdot y[n - 1] - a_2 \cdot y[n - 2])
\]

Der Wert \(y[n]\) bezeichnet den Ausgabewert zum Sample \(n\), während
\(x[n]\) den Eingangswert darstellt. Der aktuelle Ausgabewert \(y[n]\)
ergibt sich aus der gewichteten Summe der aktuellen und zwei vergangenen
Eingabewerte, subtrahiert mit den zwei gewichteten vergangen
Ausgabewerte. Die Koeffizienten \(b_0\), \(b_1\) und \(b_2\) bestimmen
den Einfluss des aktuellen und der vergangenen Eingabewerte
(Feedforward), während \(a_1\) und \(a_2\) den Rückkopplungsanteil aus
früheren Ausgabewerten (Feedback) modellieren. Im Allgemeinem wird der
Koeffizient auf \(a_0\) auf 1 normiert: \[
y[n] = b_0 \cdot x[n] + b_1 \cdot x[n - 1] + b_2 \cdot x[n - 2] - a_1 \cdot y[n - 1] - a_2 \cdot y[n - 2]
\] Zur Analyse des Frequenz- und Phasenverhaltens des Filters, wird die
Differenzengleichung mittles der z-Transformation in den z-Bereich
transformiert und daraus folgt die folgende Übertragunsfunktion.

\chapter{Übertragungsfunktion eines IIR-Filters zweiter
Ordnung}\label{uxfcbertragungsfunktion-eines-iir-filters-zweiter-ordnung}

\[
H(z) = \frac{Y(z)}{X(z)} = \frac{b_0 + b_1 z^{-1} + b_2 z^{-2}}{1 + a_1 z^{-1} + a_2 z^{-2}}
\] Bei dieser rationalen Funktion wird das Verhälnis zwischen Ausgang zu
Eingang im Frequenzbereich beschrieben. Um die Stabilität des
Biquad-Filters zu vergewissern, müssen alle Polstellen des Filters
innerhalb des Einheitskreises der z-Ebene liegen.

\section{Implementierungsstrukturen biquadratischer
Filter}\label{implementierungsstrukturen-biquadratischer-filter}

\chapter{Differenzengleichung der Direktform
1:}\label{differenzengleichung-der-direktform-1}

\(y[n] = w[n] - a_1 \cdot y[n-1] - a_2 \cdot y[n-2]\)
\(w[n] = b_0 \cdot x[n] + b_1 \cdot x[n - 1] + b_2\)

Differenzengleichung der Direktform 2:

\(y[n] = b_0 \cdot w[n] + b_1 \cdot w[n-1] + b_2 \cdot w[n-2]\)
\(w[n] = x[n] - a_1 \cdot w[n-1] - a_2 \cdot w[n-2]\)

Differenzengleichung der transponierten Direktform 2:

\(y[n] = b_0 x[n] + s_1[n-1]\)
\(s_1[n] = s_2[n - 1] + b_1 x[n] - a_1 y[n]\)
\(s_2[n] = b_2 x[n] - a_2 y[n]\)

\section{Bilineartransformation}\label{bilineartransformation}

Mittels der Bilineartransformation ist es möglich einen bestehenden
analogen Filter, anhand seiner Übertragungsfunktion zu
digitalisieren.Die analogen Biquad Filter werden aus dem Expirement 4
des Analog Systems Lab Kit PRO entnommen.

\chapter{Audio Reference Design}\label{audio-reference-design}

Hier wird erklärt wie das Audio Referenzdesign aus dem Base-Overlay
entstanden ist. Prinzipiell wurde der Audioanteil aus dem Base-Overlay
übernommen und die nicht benötigten Elemente entfernt. Die Einstellungen
der Blöcke wurden ebenfalls übernommen. Um das Design in Vivado zu
erhalten, wurde das Base-Overlay entsprechend der Anleitung auf dem
\href{https://github.com/Xilinx/PYNQ/tree/master}{Pynq-Repo} unter Linux
neu generiert. \#\#\# Einstellungen \#\#\#\# Processing System (PS): In
der Clock Config muss eines der Clock Signale auf 100 MHz eingestellt
werden, da der \emph{audio\_codec\_controller} nur 100 MHz Clock Signale
akzeptiert. Sollte ein anderer Tackt gegeben werden, gibt es eine
Warnung in Vivado. Zusätzlich muss ein ein I2C Port am PS eingeschaltet
werden. Im Base-Overlay wurde hierfür \emph{I2C1}.

\subsubsection{Constraints:}\label{constraints}

Die Constraints entstammen aus dem
\href{https://github.com/Xilinx/PYNQ/tree/master}{Pynq-Repo}. Alles, was
nicht benötigt wurde auskommentiert. Die Constraints weisen unter
anderem den Benötigten internen Pis Namen zu, damit diese besser
nachvollzogen werden können. Es ist notwendig, dass die Namen an dem
Ein- und Ausgängen aus den Constraints übernommen werden, sonst greifen
die Treiber in Pynq nicht.

\subsubsection{Clocking Wizzard:}\label{clocking-wizzard}

Die Einstellungen des Clocking Wizzard wurden aus dem Base-Overlay
übernommen. Die Aufgabe des Clocking Wizzard ist es den 10 MHz Master
Clock für den Audio Codec und damit der I2S Übertragung zu generieren.
Hier wird dieser aus dem 100 MHz Clock generiert. Dies kann auch direkt
über den PS mit den entsprechenden Einstellungen generiert werden. Das
Wichtige ist, das dieses Tackt-Signal nach außen an den richtigen Pin
geführt wird. \textbf{Zu den Einstellungen:} - Zur Sicherheit am Anfang
die Board Parameter zurücksetzen. - \emph{clk\_in1} darf \textbf{nicht}
auf \emph{sys clock} eingestellt sein! - Die Einstellungen sollten so
aussehen. - Wichtig ist, dass unten die \emph{Input Frequency(MHz)} von
\emph{clk\_in1} auf 100MHz eingestellt ist. - Wenn ein Slider vorhanden
ist, sollte dieser auf Auto gestellt werden, dann wird die Input
Frequenz übernommen. - Die restlichen Einstellungen sollten bereits so
sein. - Der \emph{clk\_out1} soll auf 10MHz eingestellt werden. - Der
\emph{Reset Type} soll auf \emph{Active Low}, da derselbe Reset vom
\emph{codec\_controller} verwendet werden soll und dieser ebenfalls
\emph{Active Low}. - Bei synchronisierten Rest kann dieser auch
\emph{Active High} sein. - Diese letzten beiden Tabs sollten dienen als
Überblick der vorherigen Einstellungen und sollten am Ende so aussehen.
- Falls nicht: Die vorherigen Einstellungen überprüfen.

\subsubsection{Audio Codec Controller:}\label{audio-codec-controller}

Der \emph{audio\_codec\_controller} ist eine eigene IP aus dem
\href{https://github.com/Xilinx/PYNQ/tree/master}{Pynq-Repo} und
generiert die vom Codec benötigten Signale sowie steuert den I2S
Daten-Stream. Die Namen des Ein- und Ausgänge sowie der IP selbst müssen
übereinstimmen, sonst greift der im Pynq enthaltene Audio-Treiber nicht
bzw. es muss sonst der neue Name an den Treiber übergeben werden. Die
\emph{Codec Address (addr1,addr0)} sollten in den Blockeinstellungen auf
\emph{11} bereits voreingestellt sein.

\subsubsection{Anmerkungen:}\label{anmerkungen}

Vivado wird Warnungen bezüglich \emph{clk\_in1} und \emph{clk\_out1}
geben, da diese Clocksignale oft durch das Führen nach außen von Vivado
als neuer Clock-Tree missinterpretiert werden, obwohl diese an dem Haubt
Clock-Tree verbunden sind. Diese Warnungen können entweder ignoriert
werden oder es kann Vivado explizit die Zusammenhänge der Clocks
mitgeteilt werden.

\section{Referenzdesign v2}\label{referenzdesign-v2}

Als Folge des Versuches Echtzeitfilterung umzusetzen entstand eine
leicht veränderte Version des ursprügliches Referenzdesign. Dieses
unterscheidet sich nur in der Art wie das 10MHz Clock Signal generiert
wird. In diesem aktualisierten Design wird das 10\,MHz-Taktsignal direkt
im Processing System (PS) erzeugt, mit einem synchronisierten
Reset-Signal. Der Vorteil dieses Ansatzes besteht darin, dass Vivado
keine Probleme mit den Clock-Trees erkennt und entsprechend auch keine
Warnungen ausgibt. Zusätzlich zum Taktsignal steht nun auch ein
passendes, synchronisiertes Reset-Signal zur Verfügung.

Das aktualisierte Design wird aller Voraussicht nach die Grundlage für
zukünftige Implementierungen bilden.




\end{document}
